\documentclass{article}

% Packages
\usepackage[ngerman]{babel} % For German language support
\usepackage[utf8]{inputenc}
\usepackage[T1]{fontenc} % For proper font encoding
\usepackage{graphicx} % For including graphics
\usepackage{multibib}
\usepackage[a4paper, margin=3cm]{geometry} % Set border to 2 cm
\usepackage{tabularx}
\usepackage{xcolor}
\usepackage{float}
\usepackage{hyperref}
\usepackage{tocloft} % Für Anpassung des Abbildungsverzeichnisses
\usepackage{amsmath} % For advanced math formatting
% \usepackage{url}

\setlength{\parindent}{0pt} % No indentation for paragraphs

% Seitenzahlen ab Inhaltsverzeichnis in römischen Zahlen
\pagenumbering{roman}

\begin{document}

% Titelseite ohne Seitenzahl
\thispagestyle{empty}
\vspace*{3cm}
\begin{center}
\textbf{\textsc{\Large{Hochschule für Technik und Wirtschaft Berlin}}}\\

\textsc{\large{Masterstudiengang Angewandte Informatik}}\\
\end{center}
\vspace{1.5cm}

\begin{figure}[!h]
\centering
\includegraphics[scale=1]{S04_HTW_Berlin_Logo_pos_FARBIG_RGB.jpg}\\\
\end{figure}
\vspace{2cm}



\begin{center}
\textbf{\Large{Low Energy Solarharvester}}\\
\end{center}



\vspace*{1cm}

\begin{center}
    \textbf{Henning, Tobias}

    \textbf{HTW Berlin}
    
    \textbf{s0578256@htw-berlin.de}
\end{center}


\begin{center}
\vfill{}
\par\end{center}

\begin{center}
\renewcommand{\today}{\ifcase \month \or Januar\or Februar\or März\or April\or Mai\or Juni\or Juli\or August\or September\or Oktober\or November\or Dezember\fi, \number \year}

\begin{center}
\large{\today}
\end{center}
\end{center}

\thispagestyle{empty}

\pagebreak{}

\renewcommand{\abstractname}{Abstract}
\begin{abstract}
Die vorliegende Arbeit beschäftigt sich mit der Entwicklung, Umsetzung und Validierung eines energieeffizienten 
Hardwareprototyps zur digitalen Darstellung von Raumbelegungsplänen an der Hochschule für Technik und Wirtschaft 
Berlin (HTW). Ausgangspunkt ist die Problematik, dass die bisherigen Papieraushänge nur zu Semesterbeginn 
aktualisiert werden und keine Möglichkeit zur Echtzeitaktualisierung bieten. Ziel der Arbeit ist es, durch den 
Einsatz von E-Ink-Displays und die Auswahl optimierter Hardwarekomponenten eine signifikante Verlängerung der 
Betriebsdauer im Vergleich zu bisherigen Prototypen zu erreichen und den Wartungsaufwand zu minimieren.

Im Rahmen einer umfassenden Analyse wurden bestehende Projekte an der HTW untersucht, deren Hardware- und 
Softwarearchitekturen verglichen und deren Laufzeiten bewertet. Darauf aufbauend erfolgte die Konzeption 
eines neuen Prototyps, bei dem insbesondere auf einen niedrigen Energieverbrauch im Deep-Sleep-Modus Wert 
gelegt wurde. Für die Umsetzung wurde ein Firebeetle 2 ESP32 Mikrocontroller ausgewählt, der durch seine 
hohe Energieeffizienz überzeugt. Ergänzend wurde ein Gehäuse entwickelt, das sich nahtlos in die bestehende 
Wandhalterung für Raumbelegungspläne der HTW einfügt und die Hardwarekomponenten zuverlässig schützt.

Die Implementierung umfasste sowohl die Anpassung der Software an die neue Hardware als auch die Fertigung 
des Gehäuses mittels Lasercutter. Die Validierung des Prototyps erfolgte durch Messungen 
der Leistungsaufnahme und Funktionstests im realen Umfeld. Die Ergebnisse zeigen, dass mit dem entwickelten 
Ansatz eine theoretische Laufzeit von etwa einem Jahr erreicht werden kann, wodurch die Anforderungen an eine 
energieeffiziente und wartungsarme Lösung erfüllt werden. Abschließend werden Optimierungspotenziale sowie 
Perspektiven für die Weiterentwicklung, wie etwa die Integration von Photovoltaikzellen, diskutiert.
\end{abstract}

\newpage


% Seitenzahlen ab hier in römischen Zahlen
\cleardoublepage
\pagenumbering{roman}

% Inhaltsverzeichnis
\tableofcontents

\newpage

% Abbildungsverzeichnis
\listoffigures

% Tabellenverzeichnis
\listoftables

% Abkürzungsverzeichnis
\section*{Abkürzungsverzeichnis}
\begin{description}
    \item[OTA] Over-the-Air
    \item[ADC] Analog-Digital-Converter 
    \item[LSF] Lehre Studium Forschung 
    \item[API] Application Programming Interface 
    \item[$V_{\text{REF}}$] Referenzspannung 
\end{description}


% Seitenzahlen ab Einleitung in arabischen Zahlen, beginnend mit 1
\clearpage
\pagenumbering{arabic}

\section{Einführung}
\subsection{Einleitung}

Im Hinblick auf die zunehmenden Anforderungen an Energieeffizienz und nachhaltige Systemkonzepte, gewinnt der Entwicklungsbedarf energieautarker 
elektronischer Systeme zunehmend an Bedeutung. 
Besonders im Bereich batteriegetriebener IoT-Anwendungen, stellt die begrenzte Lebensdauer von Energiespeichern eine zentrale Herausforderung dar. 
Wartungskosten, Batteriewechsel und ökologische Aspekte sprechen für alternative oder ergänzende Energieversorgungskonzepte.

Eine vielversprechende Möglichkeit stellt die Nutzung photovoltaischer Energiegewinnung unter Indoor-Lichtbedingungen dar. 
Moderne Dünnschicht-Solarzellen ermöglichen selbst bei vergleichsweise geringer Beleuchtungsstärke eine nutzbare Leistungsabgabe. 
Ob und in welchem Umfang diese Energie ausreicht, um ein energieoptimiertes System dauerhaft oder teilweise autark zu betreiben, 
ist jedoch stark anwendungsabhängig und erfordert eine systematische Untersuchung.

\subsection{Motivation}

Im ersten Projektsemester wurde ein erster Prototyp eines digitalen Raumbelegungsplans entwickelt, der auf einem E-Ink-Display basiert. 
Der Fokus lag dabei auf der Entwicklung eines energieeffizienten Systems, das eine lange Laufzeit ermöglicht.
In einer ersten theoretischen Analyse konnte für einen 1000 mAh 3,7 V Akku eine Laufzeit von etwa einem Jahr prognostiziert werden, 
was eine signifikante Verbesserung gegenüber bisherigen Prototypen darstellt.
Der daraus resultierende Energiebedarf lag bei etwa 70,66 mWh pro Woche, was einer mittleren Leistungsaufnahme von etwa 0,42 mWh pro Stunde entspricht. 
Eine erste Langzeitmessung über den Zeitraum vom 23.10.2025 bis zum 13.02.2026 liegt vor. Eine vorläufige Sichtung der Daten deutet darauf hin, 
dass die Laufzeit länger sein könnte als in der theoretischen Prognose erwartet.
Aufgrund dieses vielversprechenden Ergebnisses soll eine mögliche Integrierung von Photovoltaikzellen zur weiteren Verlängerung der Laufzeit 
untersucht werden.

\subsection{Zielsetzung}

Ziel dieser Arbeit ist die Entwicklung eines energieautarken Versorgungskonzepts mittels Photovoltaikzellen für den im ersten Projektsemester 
entwickelten Raumbelegungsplan und andere IoT-Anwendungen, die unter ähnlichen Bedingungen betrieben werden. Dabei liegt der Fokus auf einer 
maximalen Energieeffizienz, um die Laufzeit des Systems zu verlängern und den Wartungsaufwand zu minimieren. Das System soll dabei so konzipiert werden, 
dass es unabhängig vom Raumbelegungsplan betrieben werden kann, um eine flexible Integration in verschiedene Anwendungsbereiche zu ermöglichen. 
Im Anschluss soll geprüft werden, in welchem Umfang die Photovoltaikmodule unter normalen Innenlichtbedingungen Energie liefern und ob diese den 
Verbrauch des Systems langfristig decken oder die Batterielaufzeit messbar verlängern können.

\subsection{Methodik}
Zu Beginn soll eine umfassende Analyse der Energieverbrauchsmessungen des bisherigen Prototyps durchgeführt werden, um den tatsächlichen 
Energiebedarf zu ermitteln. Anschließend erfolgt eine Recherche zu geeigneten Photovoltaikzellen, die unter schlechten Lichtbedingungen eine 
möglichst hohe Energiegewinnung ermöglichen. Ist eine geeignete Zelle gefunden, muss ein dazu passender Laderegler ausgewählt werden. 
Dieser soll die Energie der Photovoltaikzelle effizient in den Akku einspeisen und gleichzeitig eine Überladung verhindern. Nachdem ein passendes 
System von Photovoltaikzelle und Laderegler ausgewählt wurde, muss eine kontinuierliche Messung der Stromgewinnung zwischen Laderegler und Akku umgesetzt 
werden, um die Effektivität der Energiegewinnung zu bewerten. Dazu muss ein geeigneter Strommonitoring-Chip ausgewählt und in die bestehende 
Hardware integriert werden. Abschließend soll eine Schnittstelle umgesetzt werden, die es ermöglicht, die gewonnenen Daten zur Energiegewinnung und zum 
Energieverbrauch zu überwachen und auszuwerten.

\section{Analyse}

\subsection{Laufzeitmessung des Raumbelegungsplans}
Eine Laufzeitmessung des Raumbelegungsplans wurde über den Zeitraum vom 23.10.2025 bis zum 13.02.2026 durchgeführt. Die Messung erfolgte über eine 
in Node-Red implementierte Schnittstelle, die für jede Aktualisierungsanfrage des Raumbelegungsplans die aktuelle Uhrzeit und die derzeitige Batteriespannung 
in eine Textdatei schreibt. Innerhalb dem Messzeitraum von 114 Tagen wurden 416 Aktualisierungsanfragen durchgeführt. In dieser Zeit ist die Batteriespannung 
von 4,05 V auf 3,96 V gesunken (Abbildung \ref{fig:battery_voltage}). 

\begin{figure}[H]
    \centering
    \includegraphics[width=1\textwidth]{../images/battery_voltage.png}
    \caption{Batteriespannung über die Zeit}
    \label{fig:battery_voltage}
\end{figure}

Über folgende Formel kann der anteilige Kapazitätsverlust $\Delta Q$ berechnet werden, wobei $U_\text{max}$ der maximalen Spannung, $U_\text{min}$ der minimalen Spannung, 
$U_\text{start}$ der Startspannung und $U_\text{end}$ der Endspannung entspricht:
\[
\Delta Q = \frac{U_\text{start} - U_\text{end}}{U_\text{max} - U_\text{min}}
= \frac{4.05 - 3.96}{4.2 - 3.2}
= \frac{0.09}{1.0} 
= 0.09
\]

Dies entspricht einem Kapazitätsverlust von etwa 9\% über den Zeitraum von 114 Tagen. Dadurch könnte die hochgerechnete Laufzeit des Systems bei etwa 1266 Tagen liegen, 
was etwa 3,5 Jahren entspricht. Es ist jedoch zu beachten, dass dies eine theoretische Prognose ist und die tatsächliche Laufzeit von verschiedenen Faktoren beeinflusst 
werden kann, wie z.B. der Alterung der Batterie, der Umgebungstemperatur und Selbstentladung. Jedoch deutet die Messung darauf hin, dass die Laufzeit des Systems länger 
sein könnte als in der theoretischen Prognose erwartet, was auf eine gute Energieeffizienz des Prototyps hindeutet.
Ein Erklärungsansatz für die längere Laufzeit könnte sein, dass die theoretische Prognose auf der konservativen Annahme basiert, dass die Wachzeit des Raumbelegungsplans 
bei 9 Minuten pro Woche liegt.


\newpage

\section{Erläuterung zur Nutzung von KI}

KI-Tools wurden in dieser Arbeit als assistives Werkzeug im Implementierungs- und Schreibprozess verwendet.
Im Schreibprozess diente ChatKI als Unterstützung für grobe Strukturierungsideen.
ChatGPT wurdegenutzt, um Rechtschreib- und Grammatikfehler im Fließtext zu überprüfen und Verbesserungsvorschläge zu erhalten.
Ebenfalls wurde ChatGPT im Implementierungsprozess verwendet, um Fehler im Code zu finden.
Es wurden keine Inhalte ungeprüft übernommen. 

\newpage

\addcontentsline{toc}{section}{Literaturverzeichnis}
\bibliographystyle{plain}
\bibliography{quellen}

\end{document}